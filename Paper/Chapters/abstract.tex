%- Usually read first by the reader
%- Best to actually write this last
%- Summary of what you did, your results and conclusions
%- It is not an introduction. No references
%- 300 words long

\begin{abstract}
This research project was tasked with investigating the potential gains that could be achieved from the use of \acl{GPGPU} Hardware in the process of file carving.
The hypothesis going into this research was; with the available time a search could be developed for file/headers and footers using \acs{GPGPU} hardware that could exceed the existing file carving solutions on CPU.
In a direct comparison the final result shows that data throughput went from 22.29MB/s using Scalpel to 340.94MB/s using the platform developed in this paper, Aletheia;
this was not the fastest data throughput seen by the Aletheia platform but the most impressive difference.
Further tests were then performed with the Aletheia platform, on the Linux and Windows Operating system, against multiple tools including CPU versions of the same tool.
These impressive results that were established far out shined the simple goal that was placed at the start.

The discussions were first launched into the development of the Aletheia header/footer searching platform, what areas worked as intended and what areas did not.
This was followed by how the improvements were reached within Aletheia, comparisons were made to existing tools and how they could improve with similar methods as were performed here before acknowledging improvements could continue through extended development.
Finally, discussion as to how the Aletheia platform could not only be used to further research in the field of File Carving but could also be expanded to work alongside existing work as a full file carving platform.

% WILL BE COMING BACK TO THIS AT THE END. PLACEHOLDER TEXT
% \textbf{Content:} Currently the focus for modern complex file carving methods make use of distributed computing, which can often turn expensive even for -- what can be considered -- simple setups.
% This project will investigate a new avenue of pattern recognition on large volumes of data via GPGPU technology as a comparatively cheaper means to achieve similar performance.\\
% \\\textbf{Aim:} This body of research aims to measure achievable performance gain for file carving using GPGPU methods and modern algorithms.\\
% \\\textbf{Method:} Initially, research will be conducted to establish best methods to perform string searches on GPGPU hardware.
% Followed by development of GPGPU file carving prototypes using different methods to perform string searches on storage devices.
% Finally metrics will be gathered for analysis.\\
% \\\textbf{Results:} Comparisons will then be observed between the prototypes after testing, in order to gain an understanding of which implementations perform fastest.
% Once established, the final product will then be compared to currently available tools.\\
% \\\textbf{Conclusion:} This project will demonstrate that there is an area of string searching that should be better explored in the context of digital forensics.
% It should also show that this area can provide a significant step-up in performance to currently available tools, at what can be considered a more reasonable cost.
\end{abstract}
